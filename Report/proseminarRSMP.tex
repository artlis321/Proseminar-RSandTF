\input{childdoc.def}
\childdocmain{}

%{\def\ignore{\usepackage{ws-procs9x6}}}%disable hyper at arxiv
{\def\ignore{\pdfoutput=1}}%use pdflatex at arxiv

\documentclass[11pt,a4paper,openany]{book}

%%%%%%%%%%%%%%%%%%%%%%%%%%%%%%%%%%%%%%%%%%%%
%%%%%%%%%%%%%%%%%%%%%%%%%%%%%%%%%%%%
%% Page layout 
\usepackage[a4paper,text={160mm,247mm},centering]{geometry}
% setspace does proper distances between lines, but leaves footnotes etc. intact
\usepackage{setspace}
\usepackage{rotating}
\usepackage{mathfixs}
\setstretch{1.1}

%% Fonts and languages
\usepackage[T1]{fontenc}
\usepackage[utf8]{inputenc}
\usepackage[icelandic,english]{babel}

%%%%%%%%%%%%%%%%%%%%%%%%%%%%%%%%%%%%%%%%%%%%
%%%%%%%%%%%%%%%%%%%%%%%%%%%%%%%%%%%%
%% Fancy headers, if desired
%\usepackage{fancyhdr, dsfont}
%\renewcommand{\headrulewidth}{0.4pt}
%\renewcommand{\footrulewidth}{0.4pt}

%%%%%%%%%%%%%%%%%%%%%%%%%%%%%%%%%%%%%%%%%%%%
%%%%%%%%%%%%%%%%%%%%%%%%%%%%%%%%%%%%
% Several Packages: Math, Graphics
\usepackage{amsmath, amssymb, amsfonts, geometry}
\usepackage{amsthm}

\usepackage{mathrsfs}
\usepackage{txfonts}
\usepackage{braket}

\usepackage{showlabels}

\usepackage{lipsum}
%%%%%%%%%%%%%%%%%%%%%%%%%%%%%%%%%%%%%%%%%%%%%%%%%%%%%%%%%%%%%%%%%%%%%%%%%%%%%%
%%%%%  METAPOST / Colors / Graphics
%%%%%%%%%%%%%%%%%%%%%%%%%%%%%%%%%%%%%%%%%%%%%%%%%%%%%%%%%%%%%%%%%%%%%%%%%%%%%%
\usepackage[compile=false,fonts]{mpostinl}
%\usepackage{graphbox}
%\pdfminorversion=5
%\pdfcompresslevel=9
%\pdfobjcompresslevel=3

%%%%%%%%%%%%%%%%%%%%%%%%%%%%%%%%%%%%%%%%%%%%%%%%%%%%%%%%%%%%%%%%%%%%%%%%%%%%%%
%%%%%  Define some nice colors
%%%%%%%%%%%%%%%%%%%%%%%%%%%%%%%%%%%%%%%%%%%%%%%%%%%%%%%%%%%%%%%%%%%%%%%%%%%%%%
%\usepackage{xcolor}
\usepackage[usenames,dvipsnames,table]{xcolor}
\definecolor{mygreen}{rgb}{0,0.4,0}
\definecolor{myblue}{rgb}{0,0.0,0.4}
\definecolor{refrcolor}{rgb}{0,0.4,0}
\definecolor{cgreen}{rgb}{0,0.7,0}
\definecolor{ecolor}{rgb}{.52,.03,.06}
\definecolor{bgcolor}{rgb}{.96,.95,.80}
\definecolor{bgcolordark}{rgb}{.80,.80,.67}
\definecolor{faint}{rgb}{.80,.80,.80}
\definecolor{myred}{rgb}{0.55,0.0,0.05}

%%%%%%%%%%%%%%%%%%%%%%%%%%%%%%%%%%%%%%%%%%%%%%%%%%%%%%%%%%%%%%%%%%%%%%%%%%%%%%
%%%%%  Define some nice colors
%%%%%%%%%%%%%%%%%%%%%%%%%%%%%%%%%%%%%%%%%%%%%%%%%%%%%%%%%%%%%%%%%%%%%%%%%%%%%%

%%%%%%%%%%%%%%%%%%%%%%%%%%%%%%%%%%%%%%%%%%%%%%%%%%%%%%%%%%%%%%%%%%%%%%%%%%%%%%
%%%%%  Enumitem
%%%%%%%%%%%%%%%%%%%%%%%%%%%%%%%%%%%%%%%%%%%%%%%%%%%%%%%%%%%%%%%%%%%%%%%%%%%%%%
\usepackage{enumitem}

%%%%%%%%%%%%%%%%%%%%%%%%%%%%%%%%%%%%%%%%%%%%%%%%%%%%%%%%%%%%%%%%%%%%%%%%%%%%%%
%%%%%  Hacks for particular reports.
%%%%%%%%%%%%%%%%%%%%%%%%%%%%%%%%%%%%%%%%%%%%%%%%%%%%%%%%%%%%%%%%%%%%%%%%%%%%%%

%added for report A1 to work
\usepackage[framemethod=default]{mdframed}
\newmdtheoremenv[backgroundcolor=black!10,linewidth=0pt]{definition}{Definition}[chapter]
\newmdtheoremenv[backgroundcolor=black!10,linewidth=0pt]{theorem}{Theorem}[chapter]
\newmdtheoremenv[backgroundcolor=black!10,linewidth=0pt]{lemma}{Lemma}[chapter]
%\newtheorem{definition}{Definition}
%\newtheorem{theorem}{Theorem}
%added for report A3 to work
\DeclareMathOperator{\I}{I}

%added for report B2 to work
\usepackage{wrapfig}
%added for report B3 to work
\usepackage{amsthm}
%added for report B3 to work
\usepackage{relsize}
%added for report D2 to work
\usepackage{mathtools}
\usepackage{listings}

%added for report E3 to work
%\usepackage{standalone}
%%%%%%%%%%%%%%%%%%%%%%%%%%%%%%%%%%%%%%%%%%%%%%%%%%%%%%%%%%%%%%%%%%%%%%%%%%%%%%
%%%%%  Include the METAPOST drawing setup
%%%%%%%%%%%%%%%%%%%%%%%%%%%%%%%%%%%%%%%%%%%%%%%%%%%%%%%%%%%%%%%%%%%%%%%%%%%%%%

%\input{drinfeldmp}

%%%%%%%%%%%%%%%%%%%%%%%%%%%%%%%%%%%%%%%%%%%%
%%%%%%%%%%%%%%%%%%%%%%%%%%%%%%%%%%%%
% The package "shuffle" is using old pi-type fonts which are not 
% properly treated in several modern text-processing systems. In 
% particular publishing with IOP (JPhysA) causes serious trouble. 
% The following tex-code provides a reasonable shuffle symbol. 
% \usepackage{shuffle}

\makeatletter
\providecommand*{\shuffle}{%
  \mathbin{\mathpalette\shuffle@{}}%
}
\newcommand*{\shuffle@}[2]{%
  % #1: math style
  % #2: unused
  \sbox0{$#1\vcenter{}$}%
  \kern .15\ht0 % side bearing
  \rlap{\vrule height .25\ht0 depth 0pt width 2.5\ht0}%
  \raise.1\ht0\hbox to 2.5\ht0{%
    \vrule height 1.75\ht0 depth -.1\ht0 width .17\ht0 %
    \hfill
    \vrule height 1.75\ht0 depth -.1\ht0 width .17\ht0 %
    \hfill
    \vrule height 1.75\ht0 depth -.1\ht0 width .17\ht0 %
  }%
  \kern .15\ht0 % side bearing
}
\makeatother

%%%%%%%%%%%%%%%%%%%%%%%%%%%%%%%%%%%%%%%%%%%%
%%%%%%%%%%%%%%%%%%%%%%%%%%%%%%%%%%%%
% automatic bold math in headlines
\makeatletter
\g@addto@macro\bfseries{\boldmath}
\makeatother

%%%%%%%%%%%%%%%%%%%%%%%%%%%%%%%%%%%%%%%%%%%%
%%%%%%%%%%%%%%%%%%%%%%%%%%%%%%%%%%%%
% change appearance of the table of contents
\usepackage{tocloft}
\setlength{\cftchapnumwidth}{2.4em}
\setlength{\cftsecindent}{2.5em}
\setlength{\cftsecnumwidth}{1.8em}
\setlength{\cftsubsecindent}{4.3em}
\setlength{\cftsubsecnumwidth}{2.7em}

\addto\captionsenglish{\renewcommand{\chaptername}{Report}}

%%%%%%%%%%%%%%%%%%%%%%%%%%%%%%%%%%%%%
%%%%%%%%%%%%%%%%%%%%%%%%%%%%%%%%%%%%%
% format for equation numbers
\numberwithin{equation}{chapter}
\renewcommand{\theequation}{\thechapter\hspace{0.1em}|\arabic{equation}}
\numberwithin{definition}{chapter}
\numberwithin{theorem}{chapter}
\numberwithin{lemma}{chapter}
%\renewcommand{\thedefinition}{\thechapter\hspace{0.1em}|\arabic{definition}}
%\renewcommand{\thetheorem}{\thechapter\hspace{0.1em}|\arabic{theorem}}
\renewcommand{\thedefinition}{\arabic{definition}}
\renewcommand{\thetheorem}{\arabic{theorem}}
\renewcommand{\thelemma}{\arabic{lemma}}

%%%%%%%%%%%%%%%%%%%%%%%%%%%%%%%%%%%%%%%%%%%%
%%%%%%%%%%%%%%%%%%%%%%%%%%%%%%%%%%%%
% macros for objects with two-line labels 

\usepackage{xparse}

% nice function for \omwb
\ExplSyntaxOn
\NewDocumentCommand{\Gtargz}{m m}
{
 \Gt\left(\begin{smallmatrix}
 \Gtargz_print:n {#1} \\
 \Gtargz_print:n {#2}
 \end{smallmatrix};z\right)
}
\seq_new:N \l_Gtargz_list_seq
\cs_new_protected:Npn \Gtargz_print:n #1
{
  \seq_set_split:Nnn \l_Gtargz_list_seq { , } { #1 }
  \seq_use:Nn \l_Gtargz_list_seq { , & }
}
\ExplSyntaxOff

\ExplSyntaxOn
\NewDocumentCommand{\Gtargzt}{m m}
{
 \Gt\left(\begin{smallmatrix}
 \Gtargzt_print:n {#1} \\
 \Gtargzt_print:n {#2}
 \end{smallmatrix};z,\tau\right)
}
\seq_new:N \l_Gtargzt_list_seq
\cs_new_protected:Npn \Gtargzt_print:n #1
{
  \seq_set_split:Nnn \l_Gtargzt_list_seq { , } { #1 }
  \seq_use:Nn \l_Gtargzt_list_seq { , & }
}
\ExplSyntaxOff
\newcommand{\SI}[1]{\Sel[#1]}

\ExplSyntaxOn
\NewDocumentCommand{\SIE}{m m}
{
\SelE\!\Big[\begin{smallmatrix}
 \SI_print:n {#1} \\
 \SI_print:n {#2}
 \end{smallmatrix}\Big]
}
\seq_new:N \l_SI_list_seq
\cs_new_protected:Npn \SI_print:n #1
{
  \seq_set_split:Nnn \l_SI_list_seq { , } { #1 }
  \seq_use:Nn \l_SI_list_seq { , & }
}
\ExplSyntaxOff

\usepackage[pdfencoding=auto,bookmarks=true,hyperfigures=true]{hyperref}%
\PassOptionsToPackage{unicode}{hyperref}

\usepackage{url}
%\PassOptionsToPackage{naturalnames}{hyperref}
%\usepackage{graphicx}% uncomment for latex-use
\usepackage{graphbox}
\usepackage{float}
% Collect successive citations appropriately
\usepackage[nosort]{cite}
% \usepackage[bulletsep]{collref}
% \usepackage{booktabs}
\usepackage{lmodern}
% \renewcommand{\ttdefault}{lmtt}
% \usepackage{bm}
\usepackage{amsbsy}

%\usepackage[arrow, matrix, curve]{xy}
\usepackage[font=small,labelfont=bf]{caption}


\setcounter{MaxMatrixCols}{20}
%%%%%%%%%%%%%%%%%%%%%%%%%%%%%%%%%%%%%%%%%%%%
%%%%%%%%%%%%%%%%%%%%%%%%%%%%%%%%%%%%
% box with a diagonal line for the upper left cell in a table
% \usepackage{diagbox}
% \usepackage{array}

%%%%%%%%%%%%%%%%%%%%%%%%%%%%%%%%%%%%%
%%%%%%%%%%%%%%%%%%%%%%%%%%%%%%%%%%%%%
% tiny showkeys
%\usepackage[notcite,notref]{showkeys}
%\def\showkeyslabelformat#1{{\usefont\encodingdefault\familydefault\seriesdefault
%\shapedefault
%\normalfont\tiny\ttfamily#1}}
%\def\showkeysrefformat#1{{\normalfont\tiny\ttfamily#1}}
%\makeatletter
%\def\SK@@ref#1>#2\SK@{%
% {\@inlabelfalse\leavevmode\vbox to\z@{%
% \vss\SK@refcolor\rlap{\vrule\raise .75em%
%  \hbox{\showkeysrefformat{#2}}}}}}
%\makeatother


%%%%%%%%%%%%%%%%%%%%%%%%%%%%%%%%%%%%%
%%%%%%%%%%%%%%%%%%%%%%%%%%%%%%%%%%%%%
% equations, sections, figures, tables and appendices
\newcommand{\eqn}[1]{eq.~\eqref{#1}}
\newcommand{\Eqn}[1]{Equation~\eqref{{#1}}}
\newcommand{\eqns}[2]{eqs.~\eqref{#1} and~\eqref{#2}}
\newcommand{\Eqns}[2]{Eqs.~\eqref{#1} and~\eqref{#2}}

% nice references
\newcommand{\rcite}[1]{ref.~\cite{#1}}
\newcommand{\rcites}[1]{refs.~\cite{#1}}

%%%%%%%%%%%%%%%%%%%%%%%%%%%%%%%%%%%%%
%%%%%%%%%%%%%%%%%%%%%%%%%%%%%%%%%%%%%
%arxiv links
\providecommand{\href}[2]{#2}
\newcommand{\arxivlink}[1]{\href{http://arxiv.org/abs/#1}{arxiv:#1}}

%%%%%%%%%%%%%%%%%%%%%%%%%%%%%%%%%%%%%
%%%%%%%%%%%%%%%%%%%%%%%%%%%%%%%%%%%%%
% starting times
\newcommand{\teight}{7\textsuperscript{\,\underline{45}}\,am}
\newcommand{\tnine}{9\textsuperscript{\,\underline{00}}\,am}
\newcommand{\tten}{10\textsuperscript{\,\underline{15}}\,am}

\newcommand{\tss}{\textsuperscript}
%%%%%%%%%%%%%%%%%%%%%%%%%%%%%%%%%%%%%
%%%%%%%%%%%%%%%%%%%%%%%%%%%%%%%%%%%%%
% references
\makeatletter
\def\mr@ignsp#1 {\ifx\:#1\@empty\else #1\expandafter\mr@ignsp\fi}%
\newcommand{\multiref}[1]{\begingroup%\let\protect\string%
\xdef\mr@no@sparg{\expandafter\mr@ignsp#1 \: }%
\def\mr@comma{}%
\@for\mr@refs:=\mr@no@sparg\do{\mr@comma\def\mr@comma{,}\ref{\mr@refs}}%
\endgroup}
\renewcommand{\eqref}[1]{(\multiref{#1})}
\makeatother

%%%%%%%%%%%%%%%%%%%%%%%%%%%%%%%%%%%%%%%%%%%%
%%%%%%%%%%%%%%%%%%%%%%%%%%%%%%%%%%%%
%pdf setup
\providecommand{\hypersetup}[1]{}
\providecommand{\texorpdfstring}[2]{#1}

\hypersetup{plainpages=false}
\hypersetup{pdfpagemode=UseNone}
%\hypersetup{pdfpagemode=UseOutlines}
%\hypersetup{pdfpagelabels=true}
\hypersetup{bookmarksnumbered=true}
\hypersetup{pdfstartview=FitH}
\hypersetup{colorlinks=false}
%\hypersetup{citebordercolor={.5 1 .5}}
%\hypersetup{citebordercolor={0 .9 .0}}
\hypersetup{citebordercolor={0.7 0.7 1}}
%\hypersetup{urlbordercolor={.5 1 1}}
\hypersetup{urlbordercolor={.4 .8 1}}
%\hypersetup{linkbordercolor={1 .7 .7}}
\hypersetup{linkbordercolor={1 .8 .6}}
%\hypersetup{pdfborder={0 0 1 [3]}}
\hypersetup{colorlinks=true, urlcolor=[rgb]{0.13,0.30,0.45}, linkcolor=[rgb]{0.13,0.30,0.45}, citecolor=[rgb]{0.55,0.0,0.05}}
%%%%%%%%%%%%%%%%%%%%%%%%%%%%%%%%%%%%%
%%%%%%%%%%%%%%%%%%%%%%%%%%%%%%%%%%%%%
%title data
\makeatletter
\let\@keywords\@empty
\let\@subject\@empty
\providecommand{\keywords}[1]{\gdef\@keywords{#1}}
\providecommand{\subject}[1]{\gdef\@subject{#1}}
\def\thetitle{\@title}
\def\theauthor{\@author}
\def\thesubject{\@subject}
\def\thedate{\@date}
\def\thekeywords{\@keywords}
\makeatother
\AtBeginDocument{
\hypersetup{pdftitle={\thetitle}}%
\hypersetup{pdfauthor={\theauthor}}%
\hypersetup{pdfsubject={\thesubject}}%
\hypersetup{pdfkeywords={\thekeywords}}%
}

\usepackage{chapterbib}
\usepackage{etoolbox}
\patchcmd{\thebibliography}{\chapter*}{\section*}{}{}

%%%%%%%%%%%%%%%%%%%%%%%%%%%%%%%%%%%%%
%%%%%%%%%%%%%%%%%%%%%%%%%%%%%%%%%%%%%
% Notes
%\newif\ifnote 
%\notetrue
%\newcommand{\jbnote}[1]{{\ifnote\textcolor{MidnightBlue}{\normalfont\scriptsize\sffamily
%\hspace{.1cm}JB: #1\hspace{.1cm}}\fi}} 

\allowdisplaybreaks

%%%%%%%%%%%%%%%%%%%%%%%%%%%%%%%%%%%%%%%%%%%%%%%%%%%%%%%%%%%%%%%%%%%%%%%%%%%%%%%%
% vectors and matrices
\newcommand{\colvec}[1]{\begin{pmatrix}#1\end{pmatrix}}
\newcommand{\mat}[1]{\begin{pmatrix}#1\end{pmatrix}}
\newcommand{\detmat}[1]{\begin{vmatrix}#1\end{vmatrix}}
\newcommand{\smat}[1]{\Big(\!\begin{smallmatrix}#1\end{smallmatrix}\!\Big)}

%%%%%%%%%%%%%%%%%%%%%%%%%%%%%%%%%%%%%
%%%%%%%%%%%%%%%%%%%%%%%%%%%%%%%%%%%%%
% relax the previous definition and define a new mathoperator. 
\let\Re\relax\DeclareMathOperator{\Re}{Re}
\let\Im\relax\DeclareMathOperator{\Im}{Im}

% quod erat demonstrandum
% this command is defined in amsthm
%\newcommand{\qed}
%{\begin{flushright}\ensuremath{\Box}\end{flushright}}

\newcommand{\pd}{\partial}
\newcommand{\vph}{\varphi}
\newcommand{\vth}{\vartheta}
\newcommand{\ve}{\varepsilon}
\newcommand{\na}{\vec{\nabla}}
\newcommand{\kb}{k_{B}}
\newcommand{\hb}{\hbar}
\newcommand{\al}{\alpha}
\newcommand{\be}{\beta}
\newcommand{\ga}{\gamma}
\newcommand{\de}{\delta}
\newcommand{\eps}{\epsilon}
\newcommand{\ze}{\zeta}
\newcommand{\io}{\iota}
\newcommand{\si}{\sigma}
\newcommand{\la}{\lambda}
\newcommand{\ka}{\kappa}
\newcommand{\om}{\omega}
\newcommand{\Ga}{\Gamma}
\newcommand{\De}{\Delta}
\newcommand{\Si}{\Sigma}
\newcommand{\La}{\Lambda}
\newcommand{\Om}{\Omega}
\newcommand{\Th}{\Theta}
\newcommand{\te}{\textrm}
\newcommand{\half}{\tfrac{1}{2}}
\newcommand{\ap}{\alpha'}
\newcommand{\Tr}{\textrm{Tr}}
\newcommand{\tdz}{\widetilde{z}}
\newcommand{\SL}{\mathrm{SL}}
\newcommand{\dd}{\mathrm{d}}
\newcommand{\ee}{\mathrm{e}}

% Zahlenmengen
\newcommand{\ZC}{\mathbb C}
\newcommand{\ZE}{\mathbb E}
\newcommand{\ZG}{\mathbb G}
\newcommand{\ZH}{\mathbb H}
\newcommand{\ZL}{\mathbb L} %freie Lie algebra
\newcommand{\ZN}{\mathbb N}
\newcommand{\ZP}{\mathbb P}
\newcommand{\ZQ}{\mathbb Q}
\newcommand{\ZR}{\mathbb R}
\newcommand{\ZZ}{\mathbb Z}

% calligraphic letters 
\newcommand{\CA}{\mathcal{A}}       
\newcommand{\CB}{\mathcal{B}}       
\newcommand{\CC}{\mathcal{C}}       
\newcommand{\CD}{\mathcal{D}}       
\newcommand{\CE}{\mathcal{E}}       
\newcommand{\CF}{\mathcal{F}}       
\newcommand{\CG}{\mathcal{G}} 
\newcommand{\CH}{\mathcal{H}}
\newcommand{\CI}{\mathcal{I}}
\newcommand{\CJ}{\mathcal{J}}       
\newcommand{\CK}{\mathcal{K}}
\newcommand{\CL}{\mathcal{L}}       
\newcommand{\CN}{\mathcal{N}}      
\newcommand{\CM}{\mathcal{M}}      
\newcommand{\CO}{\mathcal{O}}      
\newcommand{\CP}{\mathcal{P}}       
\newcommand{\CQ}{\mathcal{Q}}       
\newcommand{\CR}{\mathcal{R}}
\newcommand{\CS}{\mathcal{S}}
\newcommand{\CT}{\mathcal{T}}
\newcommand{\CU}{\mathcal{U}}       
\newcommand{\CV}{\mathcal{V}}
\newcommand{\CW}{\mathcal{W}}
\newcommand{\CX}{\mathcal{X}}
\newcommand{\CY}{\mathcal{Y}}       
\newcommand{\CZ}{\mathcal{Z}}       

% Parameters for amplitudes
\def\tree{\text{tree}}
\def\oneloop{\text{1-loop}}
\def\mmin{\text{min}}
\def\mmax{\text{max}}
\def\onel{{(1)}}
\def\twol{{(2)}}
\def\threl{{(3)}}
\def\fourl{{(4)}}
\def\MHV{\textrm{MHV}}
\def\MHVbar{$\overline{\hbox{MHV}}$}
\def\NMHVbar{$\overline{\hbox{NMHV}}$}
\def\QCD{\textrm{QCD}}
\def\sYM{\textrm{sYM}}
\def\YM{\textrm{YM}}
\def\SUGRA{\textrm{SUGRA}}
\def\open{\textrm{open}}
\def\closed{\textrm{closed}}
\def\reg{\text{reg}}
\def\unreg{\text{unreg}}

\newcommand{\sv}{\ensuremath{\text{sv}}}
\newcommand{\esv}{\ensuremath{\text{esv}}}

\newcommand{\binomial}{\binom}

\newcommand\mand{\qquad\textrm{and}\qquad}
\newcommand{\nn}{\nonumber}
\newcommand{\nnl}{\nonumber\\}

% Limits, sums and Integrals
\newcommand{\limninf}{\lim\limits_{n\to\infty}}
\newcommand{\limxzero}{\lim\limits_{x\to0}}
\newcommand{\limxone}{\lim\limits_{x\to1}}
\newcommand{\limxinf}{\lim\limits_{x\to\infty}}
\newcommand{\limxpi}{\lim\limits_{x\to\pi}}
\newcommand{\limxxzero}{\lim\limits_{x\to x_0}}
\newcommand{\limhzero}{\lim\limits_{h\to 0}}

\newcommand{\sumninf}{\sum\limits_{n=0}^\infty}
\newcommand{\sumnoneinf}{\sum\limits_{n=1}^\infty}
\newcommand{\suml}{\sum\limits}

\newcommand{\intab}{\int\limits_a^b}
\newcommand{\intl}{\int\limits}
\newcommand{\intC}{\int\limits_\SC}
\newcommand{\intinf}{\int\limits_{-\infty}^{\infty}}
\newcommand{\liml}{\lim\limits}
\DeclareMathOperator{\const}{const}

\DeclareMathOperator{\TL}{T}
\DeclareMathOperator{\TRE}{T^\textit{R}_\ve}
\DeclareMathOperator{\TTRE}{\tilde{T}^\textit{R}_\ve}
\DeclareMathOperator{\hF}{{\hat F}}
\DeclareMathOperator{\bhF}{{\bf{\hat F}}}
\DeclareMathOperator{\Rr}{\mathrm{Reg}}
\DeclareMathOperator{\GL}{\Gamma}
\DeclareMathOperator{\Gt}{\tilde{\Gamma}}
\DeclareMathOperator{\KN}{KN}
\DeclareMathOperator{\KNE}{KN^E}
\DeclareMathOperator{\Sel}{S}
\DeclareMathOperator{\SelE}{S^E}
\DeclareMathOperator{\Selbld}{\mathbf{S}}
\DeclareMathOperator{\SelbldE}{\mathbf{S}^E}
\DeclareMathOperator{\SelbldEw}{\mathbf{S}^E_{\mathit{w}}}
\DeclareMathOperator{\ELi}{ELi}
\DeclareMathOperator{\bC}{\mathbf{C}}
\DeclareMathOperator{\gm}{\gamma}
\DeclareMathOperator{\gmz}{\gamma_0}
\DeclareMathOperator{\ce}{\CE}
\DeclareMathOperator{\cez}{\CE_0}
\DeclareMathOperator{\zm}{\zeta}
\DeclareMathOperator{\zms}{\zeta^{\shuffle}}
\DeclareMathOperator{\omm}{\omega}
\DeclareMathOperator{\ommz}{\omm_0}
\DeclareMathOperator{\dlog}{\mathrm{dlog}}
\DeclareMathOperator{\sgn}{sgn}

\newcommand{\wmax}{{w_\text{max}}}
\newcommand{\El}{\text{E}}
\newcommand{\nol}{N}
\newcommand{\snol}{L}
\newcommand{\DAlg}{\mathfrak{u}}

\DeclareMathOperator{\GGs}{G}
\newcommand{\GGz}[1]{\GGs^0_{#1}}
\newcommand{\GG}[1]{\GGs_{#1}}

% some group theory shortcuts
\newcommand{\gSL}{\text{SL}}

% Propagator
\newcommand{\PP}{P}

% don't know what this command is for
\newcommand{\BB}{B}

\DeclareMathOperator{\EEs}{E}
\newcommand{\EEz}[1]{\EEs^0_{#1}}
\newcommand{\EE}[1]{\EEs_{#1}}

% Iterated elliptic integrals
\newcommand{\GLargz}[2]{\GL\left(\begin{smallmatrix}#1\\#2\end{smallmatrix};z\right)}
\newcommand{\GLarg}[3]{\GL\left(\begin{smallmatrix}#1\\#2\end{smallmatrix};#3\right)}
\newcommand{\GGG}[3]{\CG\left[\begin{smallmatrix}#1\\#2\end{smallmatrix};#3\right]}
\newcommand{\EMZVDatamine}{\texttt{https://tools.aei.mpg.de/emzv}}

% nice fractions
\usepackage{nicefrac}
\newcommand{\tauh}{\nicefrac{\tau}{2}}
\newcommand{\oneh}{\nicefrac{1}{2}}
\newcommand{\oned}{\nicefrac{1}{3}}
\newcommand{\onezd}{\nicefrac{2}{3}}
\newcommand{\taud}{\nicefrac{\tau}{3}}
\newcommand{\tauv}{\nicefrac{\tau}{4}}
\newcommand{\tauf}{\nicefrac{\tau}{5}}
\newcommand{\tauzf}{\nicefrac{2\tau}{5}}

% nice bar for dividing argument and the \tau variable
\newcommand{\db}{\hspace{1pt}|\hspace{1pt}}

%new math operators introduced by AK
\DeclareMathOperator{\Res}{Res}
\DeclareMathOperator{\ord}{ord}
\DeclareMathOperator{\Div}{Div}
\DeclareMathOperator{\RR}{R}
\DeclareMathOperator{\DD}{D}
\DeclareMathOperator{\DE}{\DD^E}
\DeclareMathOperator{\JJ}{J}
\DeclareMathOperator{\E}{E}
\DeclareMathOperator{\Li}{Li}
\DeclareMathOperator{\id}{id}
\DeclareMathOperator{\Vol}{Vol}
\DeclareMathOperator{\LL}{L}
\DeclareMathOperator{\FF}{F}
\newcommand{\pc}[2]{\left[#1\!:#2\!:1\right]} 
\DeclareMathOperator{\bL}{\boldsymbol{\mathrm{L}}}
\DeclareMathOperator{\bF}{\boldsymbol{\mathrm{F}}}

%%%%%%%%%%%%%%%%%%%%%%%%%%%%%%%%%%%%%
%%%%%%%%%%%%%%%%%%%%%%%%%%%%%%%%%%%%%

\makeatletter
% either have the word subsection etc. be part of the link
\newcommand{\namedref}[2]{\hyperref[#2]{#1~\ref*{#2}}}
% or rather not
%\newcommand{\namedref}[2]{#1~\hyperref[#2]{\ref*{#2}}}
\newcommand{\secref}{\@ifstar{\namedref{Section}}{\namedref{section}}}
\newcommand{\subsecref}{\@ifstar{\namedref{Subsection}}{\namedref{subsection}}}
\newcommand{\appref}{\@ifstar{\namedref{Appendix}}{\namedref{appendix}}}
\newcommand{\tabref}{\@ifstar{\namedref{Table}}{\namedref{table}}}
\newcommand{\figref}{\@ifstar{\namedref{Figure}}{\namedref{figure}}}
\newcommand{\talkref}{\@ifstar{\namedref{Talk}}{\namedref{talk}}}
\newcommand{\repref}{\@ifstar{\namedref{Report}}{\namedref{report}}}
\makeatother

%%%%%%%%%%%%%%%%%%%%%%%%%%%%%%%%%%%%%
%%%%%%%%%%%%%%%%%%%%%%%%%%%%%%%%%%%%%

% Setup for showing mathematical and conceptual difficulties as well as reading effort with using progressbar. 
\usepackage[roundnessr=0.2,subdivisions=5,width=2.42cm]{progressbar}
%
\newcommand{\triplebar}[3]{
\begin{minipage}[b]{1.6cm}
	math:\\[-1pt]
	concept: \\[-1pt]
	reading: 
\end{minipage}
\begin{minipage}[b]{2.4cm}
	\progressbar[filledcolor=RoyalBlue]{#1}\\[-1pt]
	\progressbar[filledcolor=Orange]{#2}\\[-1pt]
	\progressbar[filledcolor=OliveGreen]{#3}
\end{minipage}
}

\newcommand{\studenttutordate}[3]{
\begin{minipage}[b]{2cm}
   \textbf{Speaker:} \\[-1pt]
   \textbf{Tutor:  } \\[-1pt]
   \textbf{Date:   }    
\end{minipage}
\begin{minipage}[b]{8cm}
   #1\\[-1pt]
   #2\\[-1pt]
   #3
\end{minipage}
}

\newcommand{\speaker}[1]{\textbf{\Large{#1}}}

%%%%%%%%%%%%%%%%%%%%%%%%%%%%%%%%%%%%%
%%%%%%%%%%%%%%%%%%%%%%%%%%%%%%%%%%%%%

\setcounter{tocdepth}{3}
\setcounter{secnumdepth}{3}
\renewcommand*{\thesection}{\Alph{section}}
\renewcommand*{\thepart}{\Alph{part}}

%%%%%%%%%%%%%%%%%%%%%%%%%%%%%%%%%%%%%
%%%%%%%%%%%%%%%%%%%%%%%%%%%%%%%%%%%%%


\begin{document}
\pagestyle{plain}
%
\include{head}
\include{preliminaries}
\include{outline}
%
%%%%%%%%%%%%%%%%%%%%%%%%%%%%%%%%%%%%%%%%%%%%%%%%%%%%%%%%%%%%%%%%%%%%%%%%%%%%%%%%
\numberwithin{chapter}{part}
\renewcommand*{\thesection}{\arabic{section}}
%
% \include{reportX1}
% \include{reportX2}
% %
% Please, uncomment the line for your report here! 
% \include{reportA1}
% \include{reportA2}
% \include{reportA3}
% \include{reportA4}
% \include{reportA5}
% \include{reportA6}
% \include{reportA7}
% \include{reportA8}
% \include{reportA9}
% 
% \include{reportB1}
% \include{reportB2}
% \include{reportB3}
% \include{reportB4}
% \include{reportB5}
% \include{reportB6}
% \include{reportB7}
% \include{reportB8}
% \include{reportB9}
% \include{reportB10}
% \include{reportB11}
% Please do not delete the header data
\input{childdoc.def}
\childdocof{proseminarRSMP}
% Please do not delete the header data

%\newcommand{\mycommand}{command}
\newcommand{\bolde}[0]{\mathbf{e}}

\chapter{Theta functions, Kronecker functions and bilinear relations}
\label{rep:B12}
\speaker{Artyom Lisitsyn}

\section{Introduction}

As seen in the previous two sections (REFERENCES TO THEM), the Theta and Kronecker functions are crucial to defining multiple elliptic polylogarithms. In this section we will analyze these mathematical tools in more detail, with the goal of considering open questions in generalizations to a higher genus can be approached.

In (REF FIRST SECTION), we will start with a review of normalized holomorphic differentials on compact Riemann surfaces. These will be applied to define Abel's map, a function that takes assigns complex vectors to points on the Riemann surface, taking advantage of the differentials' properties to achieve an additively quasiperiodic result.

Then in (REF SECOND SECTION), Theta functions will discussed in detail as they can be defined at an arbitrary genus. These functions use Abel's map to assign complex numbers to a Riemann surface in a quasiperiodic way. In particular, we will find through an example that odd versions of the Theta functions are essentially elliptic/hyperelliptic analogues of the monomial $(z-z_0)$ at genus 0.

In (REF THIRD SECTION), we will define the familiar Kronecker function can be defined as a ratio of Theta functions at genus 1. This can be used to define quasiperiodic holomorphic differentials with properties including the bilinear Fay relation, which allows us to relate products of the differentials to products that may be simpler to evaluate.

Finally in (REF FOURTH SECTION), we will take a look at a particular way used today to attempt generalizations of the Kronecker function to higher genus. Schottky covers use mobius transformations to define a recursive structure on the complex plane that can be chosen to be a cover of a Riemann surface. With such covers, properties of holomorphic differentials, Abel's map, and Theta functions are easier to investigate directly, and attempts at a Kronecker function can be made.

\begin{figure}
    \center
    \includegraphics[width=0.5\textwidth]{assets/diagram.png}
    \caption{Diagramatic outline of the functions defined in this report. Dashed arrows downwards represent how concepts are used to built up to define new functions. Horizontal arrows represent inputs and outputs to the labeled objects.}
\end{figure}

\section{Abel's Map}

\subsection{Holomorphic differentials}
Holomorphic and harmonic differentials are described in detail in (REFERENCE TO A6 AND A7). However, for the readers convenience they will again be defined and their existence briefly shown here.

For more detail on the definitions and proofs shown, one may refer to \cite{Ber06}.

\begin{definition}[Holomorphic differential]
    A holomorphic differential is a smooth, complex one-form, consisting of a collection of holomorphic functions $f_\alpha$ such that
    \begin{equation}
        \omega = f_\alpha dz_\alpha
    \end{equation}
    is independent of chart.
\end{definition}

\begin{theorem}[Existence and normalization of holomorphic differentials on a compact Riemann surface]
    The dimension of the space of holomorphic differentials on a compact Riemann surface of genus $g$ is
    \begin{equation}
        \dim \mathcal H^1 = g.
    \end{equation}

    We can thus construct $g$ independent holomorphic differentials normalized such that
    \begin{equation}
        \int_{a_i} \omega_j = \delta_{ij},
    \end{equation}
    \begin{equation}
        \int_{b_i} \omega_j = \tau_{ij},
    \end{equation}
    where $\tau$ is a symmetric matrix, referred to as the period matrix.
\end{theorem}

\begin{proof}
    SKETCH OF PROOF OF THEOREM ABOVE
\end{proof}

\begin{figure}
    \center
    \includegraphics{assets/HarmonicDifferential.png}
    \caption{Figure 3.1 from \cite{Ber06}. The black curve $\gamma$ is an arbitrary cycle used to define a harmonic differential. We can define a function $f_\gamma$ that is 1 on the blue belt, 0 outside of the colored belts, and smoothly connects 0 to 1 on the green belt. The exterior derivative $df_\gamma=dh+\eta_\gamma$ lets us identify the harmonic differential $\eta_\gamma$ corresponding to the cycle $\gamma$.}
\end{figure}

\subsection{Definition of Abel's map}
\begin{definition}[Abel's map]\label{defB12:AbelMap}
    Given the normalized differentials $\omega_i$ for a Riemann surface $\mathcal M$ of genus $g$, we can define Abel's map on the fundamental domain $\mathcal L$ for some chosen basepoint $P_0$
    \begin{align}
        \mathbf{u} : & \ \mathcal L \rightarrow \mathbb C^g \\ & \ P \mapsto \begin{pmatrix}\int_{P_0}^P \omega_1 \\ \vdots \\ \int_{P_0}^P \omega_g \end{pmatrix}.
    \end{align}
    Such a mapping is well defined on the fundamental domain since the integration paths are limited to being homotopically equivalent. We can analytically continue Abel's map to beyond the fundamental domain using known results for integrations along $a$-cycles and $b$-cycles
    \begin{equation}
        \mathbf{u}(P+a_i) = \mathbf{u}(P) + \begin{pmatrix}\int_{a_i} \omega_1 \\ \vdots \\ \int_{a_i} \omega_g \end{pmatrix} = \mathbf{u}(P) + \begin{pmatrix} \delta_{i1} \\ \vdots \\ \delta_{ig} \end{pmatrix},
    \end{equation}
    \begin{equation}
        \mathbf{u}(P+b_i) = \mathbf{u}(P) + \begin{pmatrix}\int_{b_i} \omega_1 \\ \vdots \\ \int_{b_i} \omega_g \end{pmatrix} = \mathbf{u}(P) + \begin{pmatrix} \tau_{i1} \\ \vdots \\ \tau_{ig} \end{pmatrix}.
    \end{equation}
\end{definition}

The simplest example of Abel's map is one we had implicitly already seen before. At genus 1, we can use the familiar cover in which the torus is mapped to the complex plane, where the fundamental domain is a parallelogram with vertices $0$, $1$, $\tau$, $1+\tau$. There is a single holomorphic differential
\begin{equation}
    \omega = dz.
\end{equation}
Using Abel's map, with the basepoint $P_0=0$, we find that
\begin{equation}
    \mathbf{u}(z) = \int_0^z dz = z,
\end{equation}
with this result being kept as the map is continued analytically along $a$-cycles and $b$-cycles
\begin{equation}
    \mathbf{u}(P+a_i)=\mathbf{u}(z+1)=\int_0^z dz + \delta_{11} = z+1,
\end{equation}
\begin{equation}
    \mathbf{u}(P+b_i)=\mathbf{u}(z+\tau)=\int_0^z dz + \tau_{11} = z+\tau.
\end{equation}

\begin{figure}
    \center
    \includegraphics[width=0.6\textwidth]{assets/Domain.png}
    \caption{The fundamental domain at genus 1. Due to the trivial differential $\omega = dz$, we find that Abel's map at genus 1 is simply $\mathbf{u}(z)=z$, justifying the choice of cover.}
\end{figure}

Unfortunately, it is not so simple to define a fundamental domain and identify holomorphic differentials for higher genus. The resolution to this challenge is covered in (REF FOURTH SECTION).

\section{Theta functions}

With Abel's map, we have derived a additively quasiperiodic function that takes us from the manifold to vectors in $\mathbb C^g$. We seek to take advantage of this definition to find a function that instead embodies multiplicative quasiperiodicity. Then, the location of zeros on the Riemann surface will be homotopically invariant, since moving along a cycle and multiplying by some factor will not change whether the function vanishes.

The multiplicative quasiperiodicity will be facilitated by using properties of the exponential function, using the following notation to simplify some expressions
\begin{equation}
    \bolde(z) = \exp(2\pi i z), \quad \bolde(z+1)=\bolde(z).
\end{equation}

\subsection{Definition on $\mathbb C^g$}

\begin{definition}[Theta function]
    Given a $g \times g$ matrix $\tau$ that is symmetric $(\tau = \tau^T)$ and has positive definite imaginary part $(\vec n^T (\Im \tau) \vec n > 0 \forall \vec n \in \mathbb R^g \setminus \vec 0)$, we can define the associated Theta function
    \begin{equation} \label{eqnB12:ThetaGeneralDefn}
        \Theta(\vec z,\tau) := \sum_{\vec n \in \mathbb Z^g} \bolde\left(\frac{1}{2} \vec n^T \tau \vec n + \vec n^T \vec z\right)
    \end{equation}
    where $\vec z \in \mathbb C^g$. From this point forward the Theta functions explicit dependence on $\tau$ will be omitted, since the Theta functions are almost always discussed in the context of a $\tau$ chosen to be a constant.
\end{definition}

Though we will omit a formal proof thereof, it is apparent that the series will converge due to the matrix $\tau$ having a positive definite imaginary part supressing terms with large $\vec n$. As we will go on to see in the following section, the choices made for the Theta function are made to fit naturally with the formalism we developed for Riemann surfaces, as the matrix $\tau$ will be filled by the period matrix and the input $\vec z$ will be the output of Abel's map.

\begin{lemma}[Properties of the Theta function]
    \label{lemmaB12:ThetaGeneralProperties}
    The Theta function satisfies the properties
    \begin{equation} \label{eqnB12:ThetaGeneralEven}
        \Theta(\vec z) = \Theta(-\vec z)
    \end{equation}
    \begin{equation} \label{eqnB12:ThetaGeneralShift}
        \Theta(\vec z+\vec \alpha) = \Theta(\vec z) \quad \forall \vec \alpha \in \mathbb Z^g
    \end{equation}
    \begin{equation} \label{eqnB12:ThetaGeneralTauShift}
        \Theta(\vec z + \tau \vec \beta) = \bolde\left(-\frac{1}{2}\vec \beta^T \tau \vec \beta - \vec \beta^T \vec z\right) \Theta(\vec z) \quad \forall \vec \beta \in \mathbb Z^g
    \end{equation}
\end{lemma}

\begin{proof} The simple proofs of the three properties are included for the convenience of the reader.

    Proving that the Theta function is even (\ref{eqnB12:ThetaGeneralEven})
    \begin{align}
        \Theta(-\vec z) = \sum_{\vec n \in \mathbb Z^g} \bolde\left(\frac{1}{2} \vec n^T \tau \vec n + \vec n^T (-\vec z)\right) \overset{\vec n \mapsto -\vec m}{=} \sum_{\vec m \in \mathbb Z^g} \bolde\left(\frac{1}{2} \vec m^T \tau \vec m + \vec m^T \vec z\right) = \Theta(\vec z)
    \end{align}
    Proving that the Theta function is invariant under integer shifts (\ref{eqnB12:ThetaGeneralShift})
    \begin{align}
        \Theta(\vec z + \vec \alpha) &= \sum_{\vec n \in \mathbb Z^g} \bolde\left(\frac{1}{2} \vec n^T \tau \vec n + \vec n^T (\vec z + \vec \alpha)\right) \\ &=
        \sum_{\vec n \in \mathbb Z^g} \underset{=1}{\underbrace{\bolde\left(\vec n^T \vec \alpha\right)}} \bolde\left(\frac{1}{2} \vec n^T \tau \vec n + \vec n^T \vec z\right) = \Theta(\vec z)
    \end{align}
    Proving that the Theta function is invariant under period matrix shifts (\ref{eqnB12:ThetaGeneralTauShift})
    \begin{align}
        \Theta(\vec z + \tau \vec \beta) &= \sum_{\vec n \in \mathbb Z^g} \bolde\left(\frac{1}{2} \vec n^T \tau \vec n + \vec n^T (\vec z + \tau \vec \beta)\right) \\ &=
        \sum_{\vec n \in \mathbb Z^g}
        \bolde\left(-\frac{1}{2} \vec \beta^T \tau \vec \beta - \vec \beta^T \vec z\right) \bolde\left(\frac{1}{2} (\vec n+\vec \beta)^T \tau (\vec n+\vec \beta) + (\vec n+\vec \beta)^T \vec z\right)\\
        &=  \bolde\left(-\frac{1}{2} \vec \beta^T \tau \vec \beta - \vec \beta^T \vec z\right) \Theta(\vec z) 
    \end{align}
\end{proof}

\subsection{Definition on compact Riemann surface}
Now, we can combine Abel's map (\ref{defB12:AbelMap}) and the Theta function above (\ref{eqnB12:ThetaGeneralDefn}) to identify Theta functions directly on a compact Riemann surface. The properties of the two functions work perfectly together to produce desired quasiperiodic results on the Riemann surface.

\begin{definition}[Theta function on a compact Riemann surface]
    Using Abel's map $\mathbf{u} : \mathcal M \mapsto \mathbb C^g$ with some basepoint $P_0 \in \mathcal M$, and the Theta function $\Theta : \mathbb C^g \mapsto \mathbb C$ with $\tau$ corresponding to the period matrix of $\mathcal M$, we identify

    \begin{equation}
        \theta(P) = \Theta(\mathbf{u}(P),\tau)
    \end{equation}
    as a Theta function on the Riemann surface.
\end{definition}

\begin{lemma}[Properties of Theta functions on a compact Riemann surface]
    Using the analytic continuation provided by Abel's map and the quasiperiodic properties of the Theta function, we find that
    \begin{equation}
        \theta(P+a_i) = \theta(P),
    \end{equation}
    \begin{equation}
        \theta(P+b_i) = \bolde\left(-\mathbf{u}_i(P)-\frac{1}{2}\tau_{ii}\right)\theta(P).
    \end{equation}
\end{lemma}

\subsection{Characteristics and zeros}
\begin{definition}[Theta function with characteristics]
    For vectors $\epsilon,\epsilon' \in \mathbb C^g$, the Theta function with characteristics is defined as
    \begin{align}
        \Theta\begin{bmatrix} \epsilon \\  \epsilon'\end{bmatrix}(\vec z) :=& \ 
        \bolde\left(\frac{1}{8}\epsilon^T \tau \epsilon + \frac{1}{2}\epsilon^T \vec z + \frac{1}{4}\epsilon^T  \epsilon'\right)
        \Theta\left(\vec z + \frac{\epsilon'}{2}+\frac{\tau\epsilon}{2}\right)
       \\ =& \sum_{\vec n \in \mathbb Z^g} \bolde\left(\frac{1}{2} (\vec n^T \tau \vec n +\epsilon^T \tau \epsilon/4) + (\vec n+\epsilon/2)^T (\vec z+\epsilon'/2)\right).
    \end{align}

    Note that many sources (e.g. INCLUDE SOURCES HERE) use $\epsilon,\epsilon' \mapsto \epsilon/2,\epsilon'/2$ instead to simplify parts of the notation.
\end{definition}

The choice of adding characteristics to the Theta functions in this way seems arbitrary, but it actually leads to a few key properties. First, it is apparent from the definition that the Theta function, and consequently its zeros, are shifted by some controlled amount. Second, the quasiperiodicity and parity properties of the Theta function will change in ways that we will be able to take advantage of.

\begin{lemma}[Properties of the Theta functions with characteristics]
    The Theta function satisfies the properties
    \begin{equation}\Theta\begin{bmatrix}\epsilon \\ \epsilon'\end{bmatrix}(\vec z + \vec \alpha + \tau \vec \beta) =
    \ee\left(\frac{1}{2}(\epsilon^T \vec \alpha - \vec \beta^T \epsilon') - \frac{1}{2} \beta^T \tau \beta - \vec \beta \vec z\right)
    \Theta\begin{bmatrix}\epsilon \\ \epsilon'\end{bmatrix}(\vec z)
    \end{equation}

    \begin{equation}\Theta\begin{bmatrix}\epsilon + 2\eta \\ \epsilon' + 2\eta' \end{bmatrix}(\vec z) = \exp(\pi i \epsilon^T \eta')
    \Theta\begin{bmatrix}\epsilon \\ \epsilon'\end{bmatrix}(\vec z) , \quad \eta,\eta' \in \mathbb Z^g\end{equation}
    % \pause
    \begin{equation}\Theta\begin{bmatrix}\epsilon \\ \epsilon'\end{bmatrix}(-\vec z) = \exp(\pi i \epsilon^T \epsilon') \Theta\begin{bmatrix}\epsilon \\ \epsilon'\end{bmatrix}(\vec z) , \quad \epsilon,\epsilon' \in \mathbb Z^g\end{equation}

    The first property is analogous to the quasiperiodicity of the theta function before (\ref{eqnB12:ThetaGeneralShift} and \ref{eqnB12:ThetaGeneralTauShift}).
    
    The second property shows that, up to a sign, the characteristics are equivalent modulo 2.
    
    The third property defines the parity of the resulting theta function; note that it is only true for integer characteristics.
\end{lemma}

\begin{proof}
    The first and second properties are proved analogously to Lemma \ref{lemmaB12:ThetaGeneralProperties}.

    The third property uses
    \begin{equation}
        \Theta\begin{bmatrix}\epsilon \\ \epsilon'\end{bmatrix}(\vec z) = \Theta\begin{bmatrix}-\epsilon \\ -\epsilon'\end{bmatrix}(-\vec z)
    \end{equation}
    followed by applying the second property using $\eta=2\epsilon$ and $\eta=2\epsilon'$ to return to the original characteristics with a possible sign difference.
\end{proof}

The third property tells us that for characteristics $\epsilon,\epsilon' \in \mathbb Z^g$ such that $\epsilon^T\epsilon'$ is odd, the resulting Theta function will be odd. Since odd function vanish at the origin, this property can be used to identify the locations of zeros for the Theta function in general.

\begin{theorem}[Location of zeros of the Theta function]
    The zeros of the Theta function are
    \begin{equation}
        \Theta\left(\frac{\epsilon}{2} + \frac{\tau \epsilon'}{2}\right) = 0
    \end{equation}
    for all $\epsilon,\epsilon' \in \mathbb Z^g$ for which $\epsilon^T \epsilon'$ is odd.
\end{theorem}

connect to odd theta functions in B10

extend beyond presentation by pointing out that only $g$ zeros are on compact riemann surface

\subsection{Decomposition of functions}
consider talk A5 (eqn 25, 40)

link to talk A9 with divisors

\section{Kronecker function}

\subsection{Definition and properties of the Kronecker function}

\subsection{Decomposition for differentials}

\subsection{Application}

\section{Schottky Covers}

\subsection{Definition of the Schottky group and cover}
\cite{Cha22}
\cite{ComputationalSchottky}

\subsection{Differentials and Abel's map}
\cite{Cha22}
\cite{ComputationalSchottky}

\subsection{Attepmts at a Kronecker function}
\cite{Cha22}

\bibliography{reportB12}
\bibliographystyle{alphaurl}
%
%%%%%%%%%%%%%%%%%%%%%%%%%%%%%%%%%%%%%%%%%%%%%%%%%%%%%%%%%%%%%%%%%%%%%%%%%%%%%%%%
\end{document}

