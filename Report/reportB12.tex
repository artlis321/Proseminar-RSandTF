% Please do not delete the header data
\input{childdoc.def}
\childdocof{proseminarRSMP}
% Please do not delete the header data

%\newcommand{\mycommand}{command}

\chapter{Theta functions, Kronecker functions and bilinear relations}
\label{rep:B12}
\speaker{Artyom Lisitsyn}

\section{Introduction \& Background}




\cite{ZCC22}
\cite{Bertola2010RiemannSA}

\section{Decomposition of functions}

One of the goals we have as we describe complex functions on Riemann surfaces is to find a simple way to decompose any arbitrary function.

\subsection{Genus-Zero decompostion}

% lead into decomposition for genus-zero

\emph{I wrote this proof myself as an exercise. I think I am missing something in terms of how to formally treat the point at infinity; without that, my arguments that no poles $\implies$ bounded is not sound. See $\dagger$ below.}

\begin{theorem}[Decomposition for Genus-Zero]
    Let $f$ be a meromorphic function on $\mathbb{C}$ (including the point at $\infty$). Let $z_i$ be its zeros with multiplicity $n_i$ and $q_j$ be the poles with multiplicity $p_j$. Then, there exists a constant $C \in \mathbb{C}$ such that
    \begin{equation}
        f(z) = C\frac{\prod_i (z-z_i)^{n_i}}{\prod_j (z-q_j)^{p_j}} \label{eqnB12:gen0decomp}
    \end{equation}

    This is equivalent to saying that a meromorphic function is uniquely defined, up to the scaling factor $C$, by the locations and multiplicity of its zeros and poles. 
\end{theorem}

\begin{proof}
    Consider a meromorphic function $g$ with zeros and poles as described above. Let us define $h$ as $h(z) = f(z) / g(z)$. We will show that $h(z)$ must be a constant function, thus showing that $g$ is of the form desired. In order to show that $h(z)$ is constant, we can show that it is bounded and then use Liouville's Theorem [refer to reportA5? or to a source?].

    At all points besides the $z_i$ and $q_j$, we see that $f(z)$ and $g(z)$ have no zeros or poles, so $h(z)$ cannot have a pole at those locations.

    For each zero $z_i$, with multiplicity $n_i$, we can write
    \begin{align}
        f(z) &= (z-z_i)^{n_i} \tilde f(z) \\
        g(z) &= (z-z_i)^{n_i} \tilde g(z)
    \end{align}
    for some holomorphic functions $\tilde f$ and $\tilde g$ defined on a disc around $z_i$ that satisfy $\tilde f \neq 0 \neq \tilde g$.

    Then, on that disc we have $h(z) = \frac{\tilde f(z)}{\tilde g(z)}$. Since this is a ratio of two non-zero holomorphic functions, $h$ does not have a pole at $z_i$.

    Similarly, for each pole $q_j$ with multiplicity $p_j$, we can write
    \begin{align}
        f(z) &= (z-q_j)^{-p_j} \tilde f(z) \\
        g(z) &= (z-q_j)^{-p_j} \tilde g(z)
    \end{align}
    and conclude that $h$ does not have poles at $q_j$ either.

    Since $h$ has no poles, it must be bounded. \emph{$\dagger$ Some step is missing here.}
\end{proof}

\subsection{Genus-One decomposition}

\section{Main theorem}
\subsection{Preconditions}
\subsection{Boundary conditions}
\subsection{Proof}
\section{Outlook \& open questions}

\bibliography{reportB12}
\bibliographystyle{alphaurl}